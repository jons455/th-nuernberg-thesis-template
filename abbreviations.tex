% ============================================================================
% ABBREVIATIONS AND ACRONYMS
% ============================================================================
% This file defines all abbreviations used in your thesis.
%
% HOW TO USE IN YOUR THESIS:
%   Use \gls{key} in your text to insert an abbreviation.
%   Example: "The file format \gls{pdf} is widely used."
%   
%   First occurrence: Expands to "Portable Document Format (PDF)"
%   Subsequent occurrences: Shows only "PDF"
%
% HOW TO ADD NEW ABBREVIATIONS:
%   Add a line like this:
%   \newacronym{key}{ABBREVIATION}{Full Expansion}
%
% TIPS:
%   - Use lowercase for the key
%   - Keep abbreviations in UPPERCASE
%   - Make full expansions descriptive but concise
%   - Don't add acronyms you won't use in your thesis
% ============================================================================

% Some example abbreviations:

\newacronym{pdf}{PDF}{Portable Document Format}
\newacronym{html}{HTML}{Hypertext Markup Language}
\newacronym{api}{API}{Application Programming Interface}
\newacronym{gui}{GUI}{Graphical User Interface}
\newacronym{cli}{CLI}{Command-Line Interface}
\newacronym{http}{HTTP}{Hypertext Transfer Protocol}
\newacronym{db}{DB}{Database}
\newacronym{url}{URL}{Uniform Resource Locator}

%\glsaddall
\glsaddallunused