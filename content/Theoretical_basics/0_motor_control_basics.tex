\chapter{Theoretical Foundations and State of the Art}
\label{sec:theoretical_foundations}

\section{Fundamentals of Linear Control Systems}
\label{sec:control_basics}

In order to define the specific dynamics of the \gls{pmsm}, it is necessary to establish the fundamental principles of the linear control strategies employed. 
The vast majority of industrial drive systems rely on closed-loop feedback control to maintain stability and accuracy against external disturbances and parameter variations \cite{follinger_regelungstechnik_2022}.

\subsection{The PID Controller Structure}
The PID controller is the most prevalent control algorithm in automation technology due to its simple structure and robust performance across various system types \cite{zacher_regelungstechnik_2022}.

The controller calculates an actuation value $u(t)$ based on the control error $e(t)$, which is the difference between the reference setpoint $r(t)$ and the measured process variable $y(t)$:
\begin{equation}
    e(t) = r(t) - y(t)
\end{equation}

The general time-domain equation of a PID controller consists of three parallel terms:
\begin{equation}
    u(t) = \underbrace{K_p \cdot e(t)}_{Proportional} + \underbrace{K_i \int_{0}^{t} e(\tau) d\tau}_{Integral} + \underbrace{K_d \frac{d e(t)}{dt}}_{Derivative}
    \label{eq:pid_time_domain}
\end{equation}

Where $K_p$, $K_i$, and $K_d$ are the tuning coefficients. 
Each component serves a distinct role in the system dynamics:
\begin{itemize}
    \item **Proportional (P):** Reacts to the current error. A high $K_p$ increases response speed but leads to overshoot and cannot eliminate steady-state error in non-integrating plants.
    \item **Integral (I):** Reacts to the accumulation of past errors. This term is crucial for drive systems as it forces the steady-state error to zero, ensuring the actual current matches the reference exactly, even in the presence of resistance \cite{tieste_keine_2015}.
    \item **Derivative (D):** Reacts to the rate of change of the error, effectively predicting future behavior.
\end{itemize}

\subsection{PI Controllers for Motor Control}
In the context of electric current control, the full PID structure is rarely used. 
Instead, the Derivative term is typically omitted, resulting in a \gls{pi} controller.

The primary reason for excluding the D-component is the nature of the feedback signal. 
Current sensors in power electronics are subject to significant measurement noise. 
Since the derivative term amplifies high-frequency signals, differentiation of sensor noise would result in erratic and aggressive actuation signals ($u_d, u_q$). This stochastic noise would lead to excessive thermal stress on the inverter and mechanical vibrations without improving control performance \cite{schroder_elektrische_2015}.

Therefore, the discrete transfer function used in the simulation model (referred to as `PPI` in the block diagram) is a parallel PI structure:
\begin{equation}
    G_{PI}(z) = K_p + K_i \frac{T_s}{z-1}
\end{equation}
Where $T_s$ is the sampling time of the digital control system.

\subsection{Saturation and Integrator Windup}
An ideal linear controller assumes that the actuator can generate any requested voltage. 
However, real-world systems are bounded by physical limits. 

If the \gls{pi} controller requests a voltage exceeding this limit ($V_{max}$), the actuator saturates. 
If this saturation persists (e.g., during a large step change), the Integral term continues to accumulate error because the system cannot respond fast enough. 
This phenomenon is known as **Integrator Windup** \cite{zacher_regelungstechnik_2022}.

When the system finally reaches the setpoint, the integral value is excessively high, causing the controller to overshoot the target significantly while it "unwinds" the accumulated error. 
To prevent this, **Anti-Windup** strategies are implemented. 
As seen in the "Space Vector Limitation" block of the simulation, the output is clamped, and feedback is often used to halt the integration process when the limit is reached, preserving the dynamic stability of the Controller Cloning target data.