\section{Electric Drive Control}
The control of \gls{pmsm} requires a robust understanding of the machine's electromagnetic dynamics and the appropriate coordinate transformations. 
This section derives the mathematical model of the \gls{pmsm} in the synchronous rotating reference frame and details the \gls{foc} strategy used to regulate the machine's torque and flux.

\subsection{Dynamics of the Permanent Magnet Synchronous Motor}
To design an effective current controller, the dynamic behavior of the \gls{pmsm} must be mathematically described. While the physical stator quantities exist in the three-phase $abc$-system, the time-varying nature of these AC signals makes direct controller design using linear techniques—such as \gls{pi} controller regulation—challenging \cite{schroder_elektrische_2015}.

The standard approach involves transforming the three-phase system into a rotating two-phase coordinate system (d-q-frame) attached to the rotor. 
The transformation is performed using the Clarke and Park transformations. 
In this rotor-fixed frame, the AC sinusoidal quantities appear as DC values in steady-state operation, significantly simplifying the control task \cite{kellner_parameter_2012}. 

The voltage equations for a \gls{pmsm} in the \gls{dq} are given by the following differential equations:

\begin{equation}
    u_d = R_s i_d + L_d \frac{d i_d}{dt} - \omega_{el} L_q i_q
    \label{eq:ud_dynamics}
\end{equation}

\begin{equation}
    u_q = R_s i_q + L_q \frac{d i_q}{dt} + \omega_{el} L_d i_d + \omega_{el} \Psi_{PM}
    \label{eq:uq_dynamics}
\end{equation}

Where:
\begin{itemize}
    \item $u_d, u_q$ and $i_d, i_q$ are the stator voltages and currents in the $d$- and $q$-axes.
    \item $R_s$ is the stator resistance.
    \item \sloppy $L_d, L_q$ are the inductance of the direct and quadrature axes. In Interior \gls{pmsm} (IPMSM), $L_q > L_d$ due to reluctance effects \cite{ackermann_optimale_2012}.
    \item $\omega_{el}$ is the electrical angular velocity of the rotor ($\omega_{el} = p \cdot \omega_{mech}$).
    \item $\Psi_{PM}$ is the permanent magnet flux linkage.
\end{itemize}

Equations (\ref{eq:ud_dynamics}) and (\ref{eq:uq_dynamics}) reveal that the system is a Multiple-Input Multiple-Output (MIMO) system with strong internal coupling. The term $-\omega_{el} L_q i_q$ in the $d$-axis equation and $\omega_{el} L_d i_d$ in the $q$-axis equation represent the cross-coupling between the axes. Furthermore, the term $\omega_{el} \Psi_{PM}$ represents the \gls{bemf} induced by the spinning magnets \cite{kellner_parameter_2012}. 
These speed-dependent terms act as disturbances to the current control loop.

\subsection{Coordinate Transformations}
\label{sec:transformations}

The fundamental principle of \gls{foc} relies on transforming the time-variant three-phase quantities (currents and voltages) into a time-invariant rotating reference frame. This process involves two consecutive linear transformations: the Clarke transformation and the Park transformation \cite{schroder_elektrische_2015}.

\subsubsection{Clarke Transformation ($abc \to \alpha\beta$)}
The first step projects the three-phase quantities ($a, b, c$), which are spatially displaced by $120^\circ$, onto a stationary two-phase orthogonal system ($\alpha, \beta$). Assuming a balanced three-phase system where $i_a + i_b + i_c = 0$, the transformation is given by:

\begin{equation}
    \begin{bmatrix} i_\alpha \\ i_\beta \end{bmatrix} = \frac{2}{3} 
    \begin{bmatrix} 
        1 & -\frac{1}{2} & -\frac{1}{2} \\ 
        0 & \frac{\sqrt{3}}{2} & -\frac{\sqrt{3}}{2} 
    \end{bmatrix} 
    \begin{bmatrix} i_a \\ i_b \\ i_c \end{bmatrix}
    \label{eq:clarke}
\end{equation}

The scaling factor $\frac{2}{3}$ ensures amplitude invariance, meaning the magnitude of the current vector in the $\alpha\beta$-frame equals the amplitude of the phase currents.

\subsubsection{Park Transformation ($\alpha\beta \to dq$)}
The second step transforms the stationary $\alpha\beta$ coordinates into the rotating \gls{dq}. 
This transformation depends on the electrical rotor angle $\theta_{el}$, which is measured by a position sensor or estimated by an observer. 
By rotating the coordinate system synchronously with the rotor flux, the AC signals become DC quantities.

\begin{equation}
    \begin{bmatrix} i_d \\ i_q \end{bmatrix} = 
    \begin{bmatrix} 
        \cos(\theta_{el}) & \sin(\theta_{el}) \\ 
        -\sin(\theta_{el}) & \cos(\theta_{el}) 
    \end{bmatrix} 
    \begin{bmatrix} i_\alpha \\ i_\beta \end{bmatrix}
    \label{eq:park}
\end{equation}

In this frame:
\begin{itemize}
    \item The $d$-axis (direct) aligns with the permanent magnet flux.
    \item The $q$-axis (quadrature) leads the $d$-axis by $90^\circ$ electrically.
\end{itemize}

For the implementation of the control loop, the inverse transformations (Inverse Park and Inverse Clarke) are required to convert the controller's DC voltage outputs ($u_d, u_q$) back into the three-phase AC voltages ($u_a, u_b, u_c$) needed by the inverter \cite{kellner_parameter_2012}.


\subsection{Field-Oriented Control (FOC)}
\gls{foc} aims to control the \gls{pmsm} similarly to a DC motor by independently regulating the flux-generating current ($i_d$) and the torque-generating current ($i_q$).

\subsubsection{Current Control Structure}
The innermost loop of the cascade control structure is the current control loop. Classical linear control theory typically employs \gls{pi} controllers for this task due to their ability to eliminate steady-state error \cite{follinger_regelungstechnik_2022, zacher_regelungstechnik_2022}. 

The transfer function of a standard \gls{pi} controller in the discrete time domain (or continuous approximation) is:

\begin{equation}
    G_{PI}(s) = K_p \left( 1 + \frac{1}{T_i s} \right)
\end{equation}

However, a standard \gls{pi} controller assumes the plant is a simple first-order delay ($PT_1$) system. As shown in Equations (\ref{eq:ud_dynamics}) and (\ref{eq:uq_dynamics}), the \gls{pmsm} is not a simple decoupled system. 
As rotational speed $\omega_{el}$ increases, the cross-coupling terms become significant, causing the $d$-axis controller to fight against disturbances caused by the $q$-axis current, and vice versa. 
This phenomenon degrades control performance and stability at high speeds \cite{tasnim_mitigating_2024}.

\subsubsection{Decoupling Strategy}
To enable the use of independent linear \gls{pi} controllers for the $d$ and $q$ axes, a feed-forward decoupling network is implemented. 
The goal is to linearize the plant seen by the controllers by calculating and compensating for the cross-coupling voltages and \gls{bemf} \cite{li_current_2024, gemasmer_effiziente_2015}.

The required stator voltages $u_d^*$ and $u_q^*$ are calculated as the sum of the \gls{pi} controller output ($u_{PI}$) and the decoupling terms ($u_{decouple}$):

\begin{equation}
    u_d^* = \underbrace{PI(i_d^* - i_d)}_{u_{d, PI}} \underbrace{- \omega_{el} L_q i_q}_{u_{d, decouple}}
\end{equation}

\begin{equation}
    u_q^* = \underbrace{PI(i_q^* - i_q)}_{u_{q, PI}} + \underbrace{\omega_{el} L_d i_d + \omega_{el} \Psi_{PM}}_{u_{q, decouple}}
\end{equation}

By injecting these calculated decoupling voltages directly into the control output, the cross-coupling dynamics are theoretically cancelled out. 
The \gls{pi} controllers then perceive the plant simply as:
\begin{equation}
    u_{d, PI} = R_s i_d + L_d \frac{d i_d}{dt}
\end{equation}
This simplified plant allows for straightforward tuning of the controller gains ($K_p, K_i$) based on the machine's resistance and inductance, ensuring stable dynamics across the speed range \cite{tieste_keine_2015}. 
Recent research indicates that precise parameter knowledge is crucial for this decoupling to be effective. 
A parameter mismatches can lead to incomplete decoupling and residual oscillations \cite{li_current_2024}. 

\subsection{Space Vector Limitations and Voltage Constraints}
The output voltages $u_d$ and $u_q$ computed by the controller are synthesized by a voltage source inverter (VSI). 
The physical voltage available is limited by the DC-link voltage ($V_{DC}$) of the battery or power supply.

The vector magnitude of the voltage is constrained by the maximum modulation radius of the \gls{svpwm}. In the linear modulation range, this limit is typically defined as $V_{max} = \frac{V_{DC}}{\sqrt{3}}$. The constraint inequality is:

\begin{equation}
    \sqrt{u_d^2 + u_q^2} \le V_{max}
\end{equation}

If the controller demands a voltage vector that exceeds this magnitude—often occurring during high-speed operation or large transient steps—the voltage vector must be limited (scaled down) to lie on the boundary of the voltage hexagon (or circle approximation). 
Failure to limit the voltage correctly leads to "integrator windup," where the integral component of the \gls{pi} controller accumulates a large error that cannot be physically actuated, resulting in significant overshoot and settling time delays once the system returns to the linear range \cite{schroder_elektrische_2015}.

In the context of data-driven control approaches, such as Neural Networks (Controller Cloning), capturing these saturation effects is vital. 
Training data must cover the entire operating region, including voltage-limited scenarios, to ensure the learned model generalizes correctly to real-world constraints \cite{tasnim_mitigating_2024}.