\documentclass[
    11pt, % font size
    a4paper,
    sfdefaults=false,
    toc=chapterentrywithdots,
    twoside,openright,
    titlepage,
    parskip=half,
    headings=small,
    listof=totoc,
    bibliography=totoc,
    index=totoc,
    captions=tableheading,
    chapterprefix,
    listof=flat,
    final
]{scrbook}

% ============================================================================
% THESIS METADATA - CUSTOMIZE THESE VALUES FOR YOUR THESIS
% ============================================================================
% This section contains all the information that appears on your cover page,
% in the PDF metadata, and in the document headers.
% Replace the placeholder values with your own information.
% ============================================================================

\newcommand{\titel}{Prototypischer Entwurf einer LaTeX-Vorlage für Abschlussarbeiten}  % Your thesis title
\newcommand{\artderarbeit}{Masterarbeit}  % Type: Bachelorarbeit or Masterarbeit
\newcommand{\autor}{Max Mustermann}  % Your full name
\newcommand{\studiengang}{Wirtschaftsinformatik}  % Your study program (e.g., Informatik, Wirtschaftsinformatik, Medieninformatik)
\newcommand{\matrikelnr}{12345678}  % Your matriculation number (8 digits)
\newcommand{\bearbeitungsbeginn}{12.34.5678}  % Start date (DD.MM.YYYY format)
\newcommand{\bearbeitungsende}{12.34.5678}  % End date (DD.MM.YYYY format)
\newcommand{\erstgutachter}{Prof.\,Dr.~Max Mustermann}  % First examiner (primary supervisor)
\newcommand{\zweitgutachter}{Prof.\,Dr.~Max Mustermann}  % Second examiner (secondary supervisor)
\newcommand{\betreuer}{Max Mustermann}  % Advisor/Tutor from company (if thesis is company-based, otherwise remove this line)
\newcommand{\unternehmen}{Muster GmbH}  % Company name (if thesis is company-based, otherwise remove this line)
\newcommand{\logo}{official_documents/Ohm_Logo.png}  % Path to university logo image file
\newcommand{\keywords}{Schlagwort 1, Schlagwort 2, Schlagwort 3, Schlagwort 4, Schlagwort 5}  % 5-7 keywords separated by commas

% ============================================================================
% HEADER AND FOOTER CONFIGURATION
% ============================================================================
% Customize page headers and footers. Currently shows page numbers at the bottom.
% Uncomment lines below to add headers with chapter names or modify the appearance.
% ============================================================================

\usepackage[automark]{scrlayer-scrpage}
\pagestyle{scrheadings}
\chead{}
\ofoot{\pagemark}

\renewcommand*\chaptermarkformat{\chapappifchapterprefix{\ }%
  \thechapter.\enskip}

\RedeclareSectionCommand[tocindent=0pt]{section}
\RedeclareSectionCommand[tocindent=0pt]{subsection}

\usepackage{scrhack}

\usepackage[utf8]{inputenc}
\usepackage[T1]{fontenc}
\usepackage{lmodern,relsize,textcomp,csquotes}
\usepackage{amsmath,amsfonts}
\usepackage[english, ngerman]{babel} % flip first entry for main document language
\usepackage[final]{graphicx}
\usepackage[section]{placeins}
\usepackage{float}
\usepackage{setspace,geometry,xcolor}
\usepackage{makeidx}
\usepackage{paralist,ifthen,todonotes}
\usepackage{url}
\usepackage[toc]{glossaries}
\usepackage{pdfpages}
\usepackage{soul}
\usepackage{chngcntr}
\usepackage{placeins}

% ============================================================================
% CODE LISTING AND SYNTAX HIGHLIGHTING CONFIGURATION
% ============================================================================
% Configures how code blocks and listings appear in your thesis.
% Supported languages: Python, Java, C, C++, JSON, Go/Golang
% Add more language styles by creating new .sty files in listingstyles/
% ============================================================================

% Custom listingstyles
\usepackage{listings}

\lstset{
	tabsize=4,
	frame=single,
	showstringspaces=false,
	numbers=left,
	numberstyle=\ttfamily\small,
	breakautoindent=true,
  escapeinside={(*@}{@*)}
}

\usepackage{listingstyles/listings-golang}
\usepackage{listingstyles/listings-json}
\usepackage{listingstyles/listings-command}

% ============================================================================
% FIGURE AND TABLE NUMBERING AND FORMATTING
% ============================================================================
% Numbers figures and tables globally (not per chapter)
% Customizes the appearance of the list of figures and list of tables
% ============================================================================

%figure and table list format
\counterwithout{figure}{chapter}
\counterwithout{table}{chapter}
\usepackage{tocloft}
\renewcommand{\cfttabpresnum}{Tab. }
\renewcommand{\cftfigpresnum}{Abb. }
\renewcommand{\cftfigaftersnum}{:}
\renewcommand{\cfttabaftersnum}{:}
\settowidth{\cfttabnumwidth}{Tab. 10\quad}
\settowidth{\cftfignumwidth}{Abb. 10\quad}

% ============================================================================
% GLOSSARY AND ABBREVIATIONS CONFIGURATION
% ============================================================================
% Manages acronyms and abbreviations.
% Define your acronyms in abbreviations.tex file
% Use \gls{acronym_key} in your text to insert acronyms
% First use will expand to full form, subsequent uses show abbreviation only
% ============================================================================

%abbreviation list format
%\usepackage[nopostdot,nonumberlist,acronym]{glossaries}
%\usepackage[number=none]{glossaries}
%\usepackage[nonumberlist]{glossaries}
\makenoidxglossaries
\loadglsentries{abbreviations}

% ============================================================================
% TABLE FORMATTING AND CUSTOMIZATION
% ============================================================================
% Provides advanced table features:
% - longtable: Tables that span multiple pages
% - tabularx: Fixed-width tables
% - tabularray: Modern table formatting with enhanced styling
% Custom column types L and Y for left-aligned and flexible columns
% ============================================================================

% table setup
\usepackage{longtable}
\usepackage{array}
\usepackage{ragged2e}
\usepackage{lscape}
\usepackage{tabularx}
\newcolumntype{L}[1]{>{\raggedright\arraybackslash}m{#1}}
\newcolumntype{Y}{>{\raggedright\arraybackslash}m{\dimexpr\linewidth-5cm-2\tabcolsep-2\arrayrulewidth}}
\usepackage{color}
\usepackage{tabularray}


% ============================================================================
% PDF METADATA AND HYPERLINK CONFIGURATION
% ============================================================================
% Configures PDF document properties (title, author, keywords)
% and hyperlink appearance (colors, bookmarks).
% PDF metadata is populated from the THESIS METADATA section above.
% Hyperlinks are currently set to black (change colors if needed).
% ============================================================================

% pdf hyperref
\usepackage[
    bookmarks=true,
    bookmarksopen=true,
    bookmarksnumbered=true,
    bookmarksopenlevel=1,
    pdftitle={\titel},
    pdfauthor={\autor},
    pdfcreator={\autor},
    pdfsubject={\titel},
    pdfkeywords={\keywords},
    pdfpagelabels=true,
    colorlinks=true,
    linkcolor=black,
    urlcolor=black,
    anchorcolor=black,
    citecolor=black,
    filecolor=black,
    menucolor=black,
    plainpages=false,
    hypertexnames=true,
    linktocpage=false,
]{hyperref}

% ============================================================================
% PAGE LAYOUT AND SPACING CONFIGURATION
% ============================================================================
% Defines page size, margins, line spacing, and text alignment.
% Current settings:
% - Paper: A4 (210mm x 297mm)
% - Left margin: 2.5cm (binding offset)
% - Top margin: 3.0cm
% - Line spacing: 1.4 (adjust \setstretch{1.4} for different spacing)
% Modify \geometry{} values below to change margins
% ============================================================================

% page setup
% \setlength{\topskip}{\ht\strutbox}
\geometry{
  paper=a4paper,
  left=2.5cm,
  top=3.0cm,
  bindingoffset=.8cm,
}
%\onehalfspacing % line spacing default 1.5
\setstretch{1.4} % custom line spacing making lines narrower
\frenchspacing
\clubpenalty = 10000
\widowpenalty = 10000
\displaywidowpenalty = 10000

% ============================================================================
% CUSTOM COMMANDS FOR COMMON GERMAN ABBREVIATIONS
% ============================================================================
% Predefined shortcuts for frequently used German abbreviations:
% \ua = u.a. (und andere / and others)
% \zB = z.B. (zum Beispiel / for example)
% \dahe = d.h. (das heißt / that is)
% \bzw = bzw. (beziehungsweise / respectively)
% Usage: "This feature \zB allows..." -> "This feature z.B. allows..."
% ============================================================================

% some commands
\newcommand{\ua}{\mbox{u.\,a.\ }}
\newcommand{\zB}{\mbox{z.\,B.\ }}
\newcommand{\dahe}{\mbox{d.\,h.,\ }}
\newcommand{\bzw}{\mbox{bzw.\ }}
\newcommand{\bzgl}{\mbox{bzgl.\ }}
\newcommand{\eg}{\mbox{e.\,g.\ }}
\newcommand{\ie}{\mbox{i.\,e.\ }}
\newcommand{\wrt}{\mbox{w.\,r.\,t.\ }}
\newcommand{\etal}{\mbox{\emph{et.\,al.\ }}}

\begin{document}

% ============================================================================
% DOCUMENT STRUCTURE CONFIGURATION
% ============================================================================
% secnumdepth: How deep to number sections (3 = subsections get numbers)
% tocdepth: How deep to include in table of contents (2 = up to subsections)
% Adjust these values if you want different numbering/TOC structure
% ============================================================================

% headline number depth for sections and table of contents
\setcounter{secnumdepth}{3}  % numerate subsections
\setcounter{tocdepth}{2}  % ...but don't include them in toc

\frontmatter

% ============================================================================
% FRONT MATTER (FRONT PAGES: i, ii, iii, ...)
% Includes cover page, declarations, abstract, acknowledgements, TOC, etc.
% Uses roman numerals for page numbers
% ============================================================================

% ============================================================================
% THESIS COVER PAGE
% ============================================================================
% This file generates the professional cover/title page for your thesis.
% All content is pulled from metadata commands defined in thesis.tex
%
% CUSTOMIZATION:
%   - University logo: Update path in thesis.tex \logo command
%   - Faculty name: Change "Fakultät Informatik" to your faculty
%   - Company fields: Uncomment/comment "Betreuer" and "Unternehmen" lines
%                     for company-based theses (lines 30-31)
%
% VARIABLES USED (all defined in thesis.tex):
%   \logo                - Path to university logo image
%   \titel               - Your thesis title
%   \artderarbeit        - Thesis type (Bachelorarbeit/Masterarbeit)
%   \studiengang         - Study program (e.g., Informatik)
%   \autor               - Your full name
%   \matrikelnr          - Student ID number (8 digits)
%   \bearbeitungsbeginn  - Start date (DD.MM.YYYY)
%   \bearbeitungsende    - End date (DD.MM.YYYY)
%   \erstgutachter       - Primary examiner/supervisor
%   \zweitgutachter      - Secondary examiner/supervisor
%   \betreuer            - Company advisor (optional)
%   \unternehmen         - Company name (optional)
%   \the\year            - Current year (auto-generated)
%
% ============================================================================

\thispagestyle{empty}
\pdfbookmark[1]{Cover}{cov}
\begin{titlepage}

\begin{center}
\includegraphics[width=11cm]{\logo}\\[1cm]
\large{Fakultät Informatik}\\[1cm]

\large
\textbf{\titel}\\[1cm]
%
\large
\artderarbeit~im Studiengang \studiengang\\[0.5cm]
%
\large
vorgelegt von

\large
\autor\\[0.5cm]
%\small
Matrikelnummer \matrikelnr\\[0.5cm]
Bearbeitungszeitraum \bearbeitungsbeginn~bis~\bearbeitungsende
\vspace*{\fill}

\large
\begin{tabular}{p{3cm}p{8cm}}\\
Erstgutachter:  & \quad \erstgutachter\\[1.2ex]
Zweitgutachter: & \quad \zweitgutachter\\[1.2ex]
%discomment "Betreuer" and "Unternehmen" for a thesis in a company
Betreuer: & \quad \betreuer\\[1.2ex]
Unternehmen: & \quad \unternehmen
\end{tabular}
\end{center}

\begin{center}
\copyright\,\the\year
\end{center}

\vspace{-0.5cm}
\singlespacing
\small
\noindent Dieses Werk einschließlich seiner Teile ist \textbf{urheberrechtlich geschützt}.
Jede Verwertung außerhalb der engen Grenzen des Urheberrechtgesetzes ist ohne Zustimmung des Autors unzulässig und strafbar.
Das gilt insbesondere für Vervielfältigungen, Übersetzungen, Mikroverfilmungen sowie die Einspeicherung und Verarbeitung in elektronischen Systemen.



\end{titlepage}
\cleardoublepage  % Title/Cover page

% ============================================================================
% LEGAL DECLARATION (Prüfungsrechtliche Erklärung)
% ============================================================================
% CHOOSE ONE OF TWO OPTIONS BELOW:
%
% OPTION 1 - PRINT VERSION (for printing your thesis and signing by hand):
%   1. Download the PDF from the URL below
%   5. Save as: official_documents/Digital_Pruefungsrechtliche_Erklaerung_und_Erklaerung_zur_Veroeffentlichung_der_Abschlussarbeit.pdf
%   6. Uncomment the line below for print version
%
% OPTION 2 - DIGITAL VERSION (for digital submission with scanned signature):
%   1. Download the PDF from the URL below
%   2. Print it out on paper
%   3. Sign it by hand
%   4. Scan the signed document
%   5. Save as: official_documents/Digital_Pruefungsrechtliche_Erklaerung_und_Erklaerung_zur_Veroeffentlichung_der_Abschlussarbeit.pdf
%   6. Uncomment the line for digital version
%
% Download form: 
% https://www.th-nuernberg.de/fileadmin/global/Public_Docs/SB/SB_0009_FO_Pruefungsrechtliche_Erklaerung_und_Erklaerung_zur_Veroeffentlichung_der_Abschlussarbeit_public.pdf
%
% To switch between versions, comment one line out and uncomment the other:
% ============================================================================

% UNCOMMENT THIS LINE FOR PRINT VERSION (unsigned, to be printed and signed):
\includepdf{official_documents/Print_Pruefungsrechtliche_Erklaerung_und_Erklaerung_zur_Veroeffentlichung_der_Abschlussarbeit.pdf}\cleardoublepage

% UNCOMMENT THIS LINE FOR DIGITAL VERSION (scanned with original signature):
%\includepdf{official_documents/Digital_Pruefungsrechtliche_Erklaerung_und_Erklaerung_zur_Veroeffentlichung_der_Abschlussarbeit.pdf}\cleardoublepage

% paragraph setup
% identation
\setlength{\parindent}{0.5cm}
% spacing between paragraphs
\setlength{\parskip}{0pt}

\include{content/0A_abstract}\cleardoublepage  % Abstract

\include{content/0B_acknowledgements}\cleardoublepage  % Acknowledgements

\tableofcontents  % Auto-generated table of contents

% ============================================================================
% BACK MATTER (SUPPLEMENTARY PAGES: LISTS AND APPENDICES)
% ============================================================================
% Includes list of figures, tables, code listings, glossary, bibliography
% Pages are numbered but not included in the main content flow
% ============================================================================

\backmatter

\addcontentsline{toc}{chapter}{\listfigurename}  % Add to TOC
\listoffigures\cleardoublepage  % Auto-generated list of all figures

\addcontentsline{toc}{chapter}{\listtablename}  % Add to TOC
\listoftables\cleardoublepage  % Auto-generated list of all tables

\renewcommand{\lstlistlistingname}{Listingverzeichnis}
\lstlistoflistings\cleardoublepage  % Auto-generated list of code listings

\renewcommand{\glossaryname}{Abkürzungsverzeichnis}
\printnoidxglossaries\cleardoublepage  % Auto-generated glossary of abbreviations

% ============================================================================
% MAIN MATTER (MAIN CONTENT: 1, 2, 3, ...)
% ============================================================================
% The actual thesis chapters. These use standard numbering (1, 2, 3...).
% Edit the .tex files in content/ folder to add your content.
% Add or remove chapters by adding/removing \include{} lines below.
% ============================================================================

\mainmatter

\chapter{Einleitung}\label{einleitung}
Lorem ipsum dolor sit amet, consectetur adipiscing elit.
Tellus purus mollis curae himenaeos nam ac eu justo vel porta accumsan ultricies placerat cum lobortis montes fusce diam ultricies.
Accumsan rutrum montes massa habitasse quis fusce pulvinar placerat nascetur laoreet iaculis velit molestie sociosqu donec blandit praesent malesuada habitasse.
Potenti etiam porttitor maecenas sit aliquet scelerisque elementum vivamus fermentum ad montes nam fermentum in risus dui ultrices ullamcorper primis.
Donec fermentum vestibulum rutrum rhoncus pharetra congue massa blandit velit senectus torquent massa aliquet sollicitudin blandit vulputate dui egestas sed.
Varius montes adipiscing magnis parturient euismod et natoque neque ridiculus magnis ac netus aliquet turpis at mollis egestas malesuada ante.
Imperdiet adipiscing vel habitasse blandit dignissim ullamcorper malesuada ad condimentum vestibulum leo conubia suspendisse curae parturient congue vehicula maecenas amet.
Tortor consequat sed netus laoreet euismod mauris neque elementum sapien amet sem sed pretium aliquet amet lectus viverra eu imperdiet.
Praesent viverra felis iaculis vestibulum interdum conubia viverra taciti sagittis ac ridiculus enim at class ac molestie ornare senectus duis.
Vitae tempus dapibus commodo ipsum tellus senectus tempor porta pretium dictum pellentesque nullam taciti dignissim nostra enim magna quis sollicitudin.
Sollicitudin sapien magnis natoque varius at dignissim nullam etiam senectus nascetur feugiat nullam diam sagittis at bibendum habitasse vel platea.
Ornare dictumst purus eros elit pellentesque sem aenean nulla tempus ultrices suspendisse lacinia eu vulputate curae quisque lacus montes porta.

\section{Motivation}\label{einleitung_motivation}
Lorem ipsum dolor sit amet, consectetur adipiscing elit.
Tellus purus mollis curae himenaeos nam ac eu justo vel porta accumsan ultricies placerat cum lobortis montes fusce diam ultricies.
Accumsan rutrum montes massa habitasse quis fusce pulvinar placerat nascetur laoreet iaculis velit molestie sociosqu donec blandit praesent malesuada habitasse.
Potenti etiam porttitor maecenas sit aliquet scelerisque elementum vivamus fermentum ad montes nam fermentum in risus dui ultrices ullamcorper primis.
Donec fermentum vestibulum rutrum rhoncus pharetra congue massa blandit velit senectus torquent massa aliquet sollicitudin blandit vulputate dui egestas sed.
Varius montes adipiscing magnis parturient euismod et natoque neque ridiculus magnis ac netus aliquet turpis at mollis egestas malesuada ante.
Imperdiet adipiscing vel habitasse blandit dignissim ullamcorper malesuada ad condimentum vestibulum leo conubia suspendisse curae parturient congue vehicula maecenas amet.
Tortor consequat sed netus laoreet euismod mauris neque elementum sapien amet sem sed pretium aliquet amet lectus viverra eu imperdiet.
Praesent viverra felis iaculis vestibulum interdum conubia viverra taciti sagittis ac ridiculus enim at class ac molestie ornare senectus duis.
Vitae tempus dapibus commodo ipsum tellus senectus tempor porta pretium dictum pellentesque nullam taciti dignissim nostra enim magna quis sollicitudin.
Sollicitudin sapien magnis natoque varius at dignissim nullam etiam senectus nascetur feugiat nullam diam sagittis at bibendum habitasse vel platea.
Ornare dictumst purus eros elit pellentesque sem aenean nulla tempus ultrices suspendisse lacinia eu vulputate curae quisque lacus montes porta.

\section{Problemstellung}\label{einleitung_problemstellung}
Lorem ipsum dolor sit amet, consectetur adipiscing elit.
Tellus purus mollis curae himenaeos nam ac eu justo vel porta accumsan ultricies placerat cum lobortis montes fusce diam ultricies.
Accumsan rutrum montes massa habitasse quis fusce pulvinar placerat nascetur laoreet iaculis velit molestie sociosqu donec blandit praesent malesuada habitasse.
Potenti etiam porttitor maecenas sit aliquet scelerisque elementum vivamus fermentum ad montes nam fermentum in risus dui ultrices ullamcorper primis.
Donec fermentum vestibulum rutrum rhoncus pharetra congue massa blandit velit senectus torquent massa aliquet sollicitudin blandit vulputate dui egestas sed.
Varius montes adipiscing magnis parturient euismod et natoque neque ridiculus magnis ac netus aliquet turpis at mollis egestas malesuada ante.
Imperdiet adipiscing vel habitasse blandit dignissim ullamcorper malesuada ad condimentum vestibulum leo conubia suspendisse curae parturient congue vehicula maecenas amet.
Tortor consequat sed netus laoreet euismod mauris neque elementum sapien amet sem sed pretium aliquet amet lectus viverra eu imperdiet.
Praesent viverra felis iaculis vestibulum interdum conubia viverra taciti sagittis ac ridiculus enim at class ac molestie ornare senectus duis.
Vitae tempus dapibus commodo ipsum tellus senectus tempor porta pretium dictum pellentesque nullam taciti dignissim nostra enim magna quis sollicitudin.
Sollicitudin sapien magnis natoque varius at dignissim nullam etiam senectus nascetur feugiat nullam diam sagittis at bibendum habitasse vel platea.
Ornare dictumst purus eros elit pellentesque sem aenean nulla tempus ultrices suspendisse lacinia eu vulputate curae quisque lacus montes porta.

\section{Zielsetzung der Arbeit}\label{einleitung_zielsetzung_der_arbeit}
Lorem ipsum dolor sit amet, consectetur adipiscing elit.
Tellus purus mollis curae himenaeos nam ac eu justo vel porta accumsan ultricies placerat cum lobortis montes fusce diam ultricies.
Accumsan rutrum montes massa habitasse quis fusce pulvinar placerat nascetur laoreet iaculis velit molestie sociosqu donec blandit praesent malesuada habitasse.
Potenti etiam porttitor maecenas sit aliquet scelerisque elementum vivamus fermentum ad montes nam fermentum in risus dui ultrices ullamcorper primis.
Donec fermentum vestibulum rutrum rhoncus pharetra congue massa blandit velit senectus torquent massa aliquet sollicitudin blandit vulputate dui egestas sed.
Varius montes adipiscing magnis parturient euismod et natoque neque ridiculus magnis ac netus aliquet turpis at mollis egestas malesuada ante.
Imperdiet adipiscing vel habitasse blandit dignissim ullamcorper malesuada ad condimentum vestibulum leo conubia suspendisse curae parturient congue vehicula maecenas amet.
Tortor consequat sed netus laoreet euismod mauris neque elementum sapien amet sem sed pretium aliquet amet lectus viverra eu imperdiet.
Praesent viverra felis iaculis vestibulum interdum conubia viverra taciti sagittis ac ridiculus enim at class ac molestie ornare senectus duis.
Vitae tempus dapibus commodo ipsum tellus senectus tempor porta pretium dictum pellentesque nullam taciti dignissim nostra enim magna quis sollicitudin.
Sollicitudin sapien magnis natoque varius at dignissim nullam etiam senectus nascetur feugiat nullam diam sagittis at bibendum habitasse vel platea.
Ornare dictumst purus eros elit pellentesque sem aenean nulla tempus ultrices suspendisse lacinia eu vulputate curae quisque lacus montes porta.

\section{Forschungsfragen}\label{einleitung_forschungsfragen}
Lorem ipsum dolor sit amet, consectetur adipiscing elit.
Tellus purus mollis curae himenaeos nam ac eu justo vel porta accumsan ultricies placerat cum lobortis montes fusce diam ultricies.
Accumsan rutrum montes massa habitasse quis fusce pulvinar placerat nascetur laoreet iaculis velit molestie sociosqu donec blandit praesent malesuada habitasse.
Potenti etiam porttitor maecenas sit aliquet scelerisque elementum vivamus fermentum ad montes nam fermentum in risus dui ultrices ullamcorper primis.
Donec fermentum vestibulum rutrum rhoncus pharetra congue massa blandit velit senectus torquent massa aliquet sollicitudin blandit vulputate dui egestas sed.
Varius montes adipiscing magnis parturient euismod et natoque neque ridiculus magnis ac netus aliquet turpis at mollis egestas malesuada ante.
Imperdiet adipiscing vel habitasse blandit dignissim ullamcorper malesuada ad condimentum vestibulum leo conubia suspendisse curae parturient congue vehicula maecenas amet.
Tortor consequat sed netus laoreet euismod mauris neque elementum sapien amet sem sed pretium aliquet amet lectus viverra eu imperdiet.
Praesent viverra felis iaculis vestibulum interdum conubia viverra taciti sagittis ac ridiculus enim at class ac molestie ornare senectus duis.
Vitae tempus dapibus commodo ipsum tellus senectus tempor porta pretium dictum pellentesque nullam taciti dignissim nostra enim magna quis sollicitudin.
Sollicitudin sapien magnis natoque varius at dignissim nullam etiam senectus nascetur feugiat nullam diam sagittis at bibendum habitasse vel platea.
Ornare dictumst purus eros elit pellentesque sem aenean nulla tempus ultrices suspendisse lacinia eu vulputate curae quisque lacus montes porta.

\section{Aufbau der Arbeit}\label{einleitung_aufbau_der_arbeit}
Lorem ipsum dolor sit amet, consectetur adipiscing elit.
Tellus purus mollis curae himenaeos nam ac eu justo vel porta accumsan ultricies placerat cum lobortis montes fusce diam ultricies.
Accumsan rutrum montes massa habitasse quis fusce pulvinar placerat nascetur laoreet iaculis velit molestie sociosqu donec blandit praesent malesuada habitasse.
Potenti etiam porttitor maecenas sit aliquet scelerisque elementum vivamus fermentum ad montes nam fermentum in risus dui ultrices ullamcorper primis.
Donec fermentum vestibulum rutrum rhoncus pharetra congue massa blandit velit senectus torquent massa aliquet sollicitudin blandit vulputate dui egestas sed.
Varius montes adipiscing magnis parturient euismod et natoque neque ridiculus magnis ac netus aliquet turpis at mollis egestas malesuada ante.
Imperdiet adipiscing vel habitasse blandit dignissim ullamcorper malesuada ad condimentum vestibulum leo conubia suspendisse curae parturient congue vehicula maecenas amet.
Tortor consequat sed netus laoreet euismod mauris neque elementum sapien amet sem sed pretium aliquet amet lectus viverra eu imperdiet.
Praesent viverra felis iaculis vestibulum interdum conubia viverra taciti sagittis ac ridiculus enim at class ac molestie ornare senectus duis.
Vitae tempus dapibus commodo ipsum tellus senectus tempor porta pretium dictum pellentesque nullam taciti dignissim nostra enim magna quis sollicitudin.
Sollicitudin sapien magnis natoque varius at dignissim nullam etiam senectus nascetur feugiat nullam diam sagittis at bibendum habitasse vel platea.
  % Chapter 1: Introduction

\include{content/2_related_work}  % Chapter 2: Related Work

\include{content/3_theoretical_background}  % Chapter 3: Theoretical Background

\include{content/4_methodology}  % Chapter 4: Methodology

\include{content/5_results}  % Chapter 5: Results

\include{content/6_contributions}  % Chapter 6: Contributions

\include{content/7_discussion}  % Chapter 7: Discussion

\include{content/8_conclusion}  % Chapter 8: Conclusion

% ============================================================================
% BIBLIOGRAPHY AND APPENDICES
% ============================================================================
% Bibliography: Generated from refs.bib using the specified style (wmaainf)
% Appendices: Optional additional chapters/content
% ============================================================================

\bibliographystyle{wmaainf}  % Bibliography style
\bibliography{refs}\cleardoublepage  % Load bibliography from refs.bib

\appendix  % Start appendix section (uses letters: A, B, C instead of 1, 2, 3)
\chapter{<Platzhalter 10>: <Platzhalter 11>}\label{appendix_platzhalter_10_11}

\chapter{<Platzhalter 12>: <Platzhalter 13>}\label{appendix_platzhalter_12_13}

\chapter{<Platzhalter 14>: <Platzhalter 15>}\label{appendix_platzhalter_14_15}
  % Appendix content

% ============================================================================
% END OF DOCUMENT
% ============================================================================
% Document compilation ends here.
% When compiled with pdflatex/latexmk, you'll get thesis.pdf
% ============================================================================

\end{document}